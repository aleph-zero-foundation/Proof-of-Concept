\documentclass[a4paper,10pt]{article}
\usepackage[utf8]{inputenc}
\usepackage{latexsym}
\usepackage{amsmath}
\usepackage{amsthm}
\usepackage{amsfonts}
\usepackage{enumerate}
\usepackage{color}
\usepackage{listings}
\usepackage{todonotes}
\lstset{basicstyle=\ttfamily,language=Python}

\title{PoC internal report}
\author{The Aleph Zero team}
\begin{document}
 \maketitle
	This is the internal report from the final round of tests we ran on the proof-of-concept implementation of the Aleph protocol.
	The implementation is written in Python and can be accessed at \todo[inline]{Make a tag and link it here.}.
	\section{General setup}
	This sections gives an overview of the main experimental setup.
		\subsection{Settings}
			We experimented with various constants than can be set in the implementation. Short descriptions of them follow.
			\begin{description}
				\item[create delay] The minimum amount of time, in seconds, to wait between finishing a unit and initiating the creation of a new one. Values are $.5$, $1$, and $2$.
				\item[sync delay] The minimum amount of time, in seconds, to wait between initiating a sync and initiating another one. Values are between $.015625$ and $.25$, doubling each step.
				\item[committee size] The number of processes in the committee. Values are $32$, $64$, $128$, $256$ and $512$.
				\item[transactions per unit] The number of transactions included in a single unit. Values are $1$, $100$, and $1000$.
				\item[maximum number of parents] The maximal number of parents a unit can have. Values are $10$, $20$ and $128$, the latter simulating a lack of limit.
				\item[maximum number of incoming/outgoing syncs] The maximal number of incoming and outgoing syncs happening at the same time.
					These are separate constants but in our tests they share the same value. Values are $10$ and $32$.
			\end{description}
		\subsection{Main values}
			The core values in the results are outlined below.
			\begin{description}
				\item[latency] How long it took for a newly created unit to be validated. Perhaps the most important number.
				\item[transactions per second] How many transactions were introduced into the system every second.
				\item[sync delay] How long it actually took to initiate a new sync after the previous one. Important for assessing whether the processes ``keep up''.
				\item[create delay] How long it actually took to initiate a new unit creation after the previous one. Important for assessing whether the processes ``keep up''.
				\item[number of parents] How many parents did a unit have. Important for assessing whether the processes have enough information when creating units.
			\end{description}
	\section{Algorithms}
		A single process does many things:
		\begin{itemize}
			\item Creating units.
			\item Adding units to the poset.
			\item Sending and receiving units.
			\item Computing decisions for timing units.
		\end{itemize}
		A short description of the algorithms involved and their complexity follows.
		\subsection{Expensive operations}
		 Before we detail the algorithms we point out several important optimization details that we will refer to in further parts of this section.
			For purposes of complexity estimation let $N$ denote the number of processes in the committee and $p$ the number of parents a single unit has.
			The latter might vary per unit, but we usually have an upper bound.

			The most important feature of the poset we maintain in the algorithm is of course its order.
			For comparing units we use the \lstinline{below} method. In the absence of forks (which is the case in all the tests) this method works in constant time.
			This, however, requires maintaining a helper structure in every unit $U$ called \lstinline{floor}.
			This structure is a list of lists, a list at position $k$ contains the maximal units made by process $k$ that are below $U$.
			To check whether $V$ is below $U$ we check if \lstinline{U.floor[V.creator_id][0].height >= V.height}.
			Maintaining \lstinline{floor} requires more work. When adding a unit to the poset we combine the floors of its parents.
			This operation runs in $O(N*p)$ time and is one of the more expensive ones.
		\subsection{Creating units}
			The most expensive part of creating a unit $U$ is choosing its parents. Because of historical reasons we do this somewhat suboptimally.
			We set the first parent to the last unit created by the process creating $U$. Every subsequent parent we choose as follows.
			A list is created by taking all the maximal units of the poset and finding the ones which \emph{do not} satisfy the expand primes rule (described in section \ref{sec:adding}).
			We then take the list of maximal units and remove the members of the previous list and all the previous parents.
			We choose a random unit from this list and add it as a parent. This is repeated until we either cannot add any more parents or reach the limit of parents.

			\todo[inline]{Make sure we don't use greedy for any experiments or describe it here.}

			Checking whether a unit satisfies the expand primes rule takes $O(N)$ time and we have to do this for at most $N$ units.
			The whole process is repeated $p$ times, giving us a final time of $O(N^2*p)$.
			This is relatively slow, but it is worth pointing out that creation happens relatively rarely, so it shouldn't have big effects on the overall run time.
		\subsection{Adding units to the poset}
		 \label{sec:adding}
		 Adding units to the poset consists of several potentially computationally expensive steps:
			\begin{enumerate}
				\item Checking unit compliance.
				\item Computing the level of a unit.
			\end{enumerate}

			Compliance checking consists of multiple parts, but in this document we restrict the discussion to the ones that might be computationally expensive.
			For a more through list of steps see the \lstinline{check_compliance} documentation.
			\begin{enumerate}
				\item We check whether the unit proves that its creator produced a fork. This requires combining floors, which takes $O(N*p)$ time.
				\item We check whether any of its parents proves that the creator of any other parent is a forker. This takes $O(p^2)$ time,
					because the parents are already in the poset, so no floor combining is needed.
				\item We check whether the ``expand primes'' rule is satisfied. This should take at most $O(N*p)$ time.
			\end{enumerate}
			All in all compliance should take about $O(N*p)$ time, which at worst can degenerate to $O(N^2)$ if we don't restrict the number of parents.

			Computing the level consists of iterating over all prime units of the level of the unit's predecessor and checking whether a sufficient number of them is below the unit.
			It takes only about $O(N)$ time, but if the number of parents is restricted this is comparable to compliance checking.
		\subsection{Sending and receiving units}
		 Determining which units to send only takes time linear with respect to the number of units that actually have to be sent.
			Then we have to order the units topologically -- we do this before sending them, but it has to be done at some point so there is no time to be saved here.
			This takes at most $O(k*p)$ time, where $k$ is the number of units to send.

			This is mostly relevant when $k$ is big, i.e. when the processes syncing are significantly desynchronised.
			This sometimes happens in our tests, because the machines don't start at once.
			However after a while the effects are negligible.
		\subsection{Computing decisions for timing units}
		 The most time expensive part of timing decisions is checking for popularity proofs.
			To check if a unit proves the popularity of another we traverse the poset graph between the two units and check whether
			we encounter units created by sufficiently many processes. This takes time linear with the number of edges encountered,
			which is $p$, times the number of units encountered. This is only ran between units differing by $3$ levels, so the number of units
			is about $3*N*t$, where $t$ is the average number of units created by a single process at a single level (the ``thickness'' of a level).
			It is worth pointing out that in many of our experiments	$t = 1$.

			The attempt at proving popularity can be ran for up to $N$ unit consecutively, so this single operation can take $O(N*N*p*t)$ time.
			When $p$ is close to $N$ this turns into $O(N^3*t)$, which is much more expensive than any other single operation we use.
			However, we memoize the results, so this should only happen very infrequently, at most for one unit per level in the whole poset.

			Other methods relating to timing decisions take about $O(N)$ time, so much faster.
\end{document}
